\documentclass{beamer}
\usepackage[utf8]{inputenc}
\usepackage{physics}
\usepackage{listings} % Required for insertion of code
\usepackage{amsmath}
\usepackage{amsfonts}
\usepackage{amsthm}
\usepackage{amssymb}
\usepackage{mathrsfs}
\usepackage{enumitem}
\usepackage{bm}

\usetheme{Madrid}
\usecolortheme{default}
\useinnertheme{circles}
\usepackage{float}
\usepackage[backend=biber,style=numeric,sorting=none]{biblatex}
\addbibresource{refs.bib}
\definecolor{Logo1}{rgb}{0.208, 0.2865, 0.373}
\definecolor{Logo2}{rgb}{0.000, 0.674, 0.863}

\setbeamercolor*{palette primary}{bg=Logo1, fg=white}
\setbeamercolor*{palette secondary}{bg=Logo1, fg=white}
\setbeamercolor*{palette tertiary}{bg=white, fg=Logo1}
\setbeamercolor*{palette quaternary}{bg=Logo1,fg=white}
\setbeamercolor{structure}{fg=Logo1} % itemize, enumerate, etc
\setbeamercolor{section in toc}{fg=Logo1} % TOC sections

%------------------------------------------------------------
%This block of code defines the information to appear in the
%Title page
\title[Curso: Propagadoress de Polarización] %optional
{Matriz densidad reducida}

%\subtitle{And its subtitle}

\author[Daniel F. E. Bajac] % (optional)
{Daniel F. E. Bajac}


\institute[] % (optional)
{
  Instituto de Modelado e Innovación Tecnológica, CONICET\\
  Departamento de Física, Facultad de Ciencias Exactas y Naturales, UNNE
}

\date[] % (optional)
{Trabajo final del curso Propagadores de Polarización}

\titlegraphic{\includegraphics[width=2cm]{logo_imit_2.png}\hspace*{0cm}}%
  
%End of title page configuration block
%------------------------------------------------------------



%------------------------------------------------------------
%The next block of commands puts the table of contents at the 
%beginning of each section and highlights the current section:

\AtBeginSection[]
{
  \begin{frame}
    \frametitle{Table of Contents}
    \tableofcontents[currentsection]
  \end{frame}
}
%------------------------------------------------------------

\setbeamerfont{footnote}{size=\tiny}
\begin{document}

%The next statement creates the title page.
\frame{\titlepage}


%---------------------------------------------------------
%This block of code is for the table of contents after
%the title page
\begin{frame}
\frametitle{Table of Contents}
\tableofcontents

\end{frame}
%---------------------------------------------------------


\section{Definición del Problema}

%---------------------------------------------------------
%Changing visivility of the text
\begin{frame}
\frametitle{Funcional Generatriz}

En el año 2014, \footfullcite[]{QFT}
se definió por primera vez una funcional generatriz para el formalismo
de Propagadores de Polarización utilizando formalismo de superoperadores 
\pause
\begin{equation*}
	Z_{[0]} = \int D| \bm{\widetilde{h}} ) D (\bm{h}|  e^{| \bm{\widetilde{h}} ) (\bm{h}| E \hat{I} - \hat{H}_0 | \bm{\widetilde{h}} )   (\bm{h}|}
\end{equation*}

\vfill
está escrita en función del Propagador Principal, 
contiene toda la información física que surge debido a la transmisión
de los efectos de dos perturbaciones externas a través del marco electrónico 
del sistema cuántico



\end{frame}


\begin{frame}
\small
  
  \begin{equation*}\label{Z}
    Z = \int D| \bm{\widetilde{h}} ) D (\bm{h}|  e^{| \bm{\widetilde{h}} ) (\bm{h}| E \hat{I} - \hat{H}_0 | \bm{\widetilde{h}} )   (\bm{h}|}
  \end{equation*}

donde $E$ es la energía del sistema y $H_0$ es el Hamiltoniano del sistema sin perturbar, $\bm{h}$ es un conjunto de operadores 
completo con el cual se puede generar todos los estados excitados de un sistema molecular N-electrónico, 
$\bm{h} | \bm{0} \rangle = |\bm{n}\rangle$  y el estado de referencia
es un campo autoconsistente SCF.


\pause
\vfill

La integral sobre $D|\bm{h})$ significa que la exponencial tomará todos los caminos posibles, 
lo que en el caso del presenta trabajo, se tienen excitaciones de MOs, se representa con una sumatoria 
sobre todos los caminos posibles%, es decir, los ${i,j}$ pueden tomar el lugar de cualquier estado ocupado del sistema, 
%y los ${a,b}$, el lugar de cualquier estado desocupado

%\begin{equation*}\label{Z_sum}
%	Z = \sum_{ia}  e^{| \bm{\widetilde{h}}_2 ) (\bm{h}_2| E \hat{I} - \hat{H}_0 | \bm{\widetilde{h}}_2 )   (\bm{h}_2|}
%\end{equation*}

\end{frame}

\begin{frame}
  \frametitle{Operador densidad}

  Ésta función generatriz es análoga a la función de partición de la termodinámica estadística  \footfullcite[]{functional}
($Z = Tr (e^{\beta \hat{H}})$) \\ con la cual el operador densidad  del sistema se puede escribir


\begin{equation*}
	\rho = \frac{e^{| \bm{\widetilde{h}} ) (\bm{h}| E \hat{I} - \hat{H}_0 | \bm{\widetilde{h}} )   (\bm{h}|}}
	{Z}
\end{equation*}

propuesto por primera vez en el año 2018 \footfullcite[]{Millan}, para calcular el entrelazamiento entre excitaciones virtuales 
en un sistema molecular N-electrónico.


\end{frame}

\begin{frame}
  En el formalismo de superoperadores, es importante definir el conjunto de operadores completo:

\begin{equation*}
	\bm{h} =    \left\{ \bm{h}_2, \bm{h}_4, \cdots \right\} 
\end{equation*}

A los fines del presente trabajo, se considera tratar la matriz densidad a nivel de teoría RPA, 
que implica considerar al estado de referencia un SCF 
y truncar el conjunto de operadores en excitaciones simples \footfullcite[]{Revieww}  $ \bm{h} = \bm{h}_2 $ 


\begin{equation*}
	\bm{h}_2 = \left\{a^\dagger_a a_i, a^\dagger_i a_a, \cdots\right\}
\end{equation*}

donde $a^\dagger_i$ es un operador creación en el orbital ocupado $i$ y 
$a_a$ es un operador aniquilación en el orbital virtual $a$. 
%Considerar el siguiente nivel del conjunto de operadores significa considerar excitaciones dobles $\bm{h}_4$, 
%además de utilizar un estado de referencia con estados doblemente excitados.
  

Los operadores $ \bm{h}_2 $ y $ \bm{\widetilde{h}}_2 $ son operadores 
fila y columna respectivamente. Además, el producto interno entre dos operadores está definido como 

\begin{equation*}
	(\hat{P}|\hat{Q}) = \langle 0 | [\hat{P}^\dagger ,\hat{Q}] |0 \rangle
\end{equation*}

\end{frame}

\begin{frame}
  \frametitle{Matriz densidad reducida}
  El objetivo de este trabajo es \textit{Verificar que la matriz densidad $\rho$ reducida a ciertas excitaciones, 
  es decir, el operador densidad que corresponde a dichas excitaciones $\rho_{ia,jb}$, se puede escribir como}

  \begin{equation*}
  \rho_{ia,jb} =  \frac{e^{| \bm{\widetilde{h}}_{ia} ) (\bm{h}_{ia}| E \hat{I} - \hat{H}_0 | \bm{\widetilde{h}}_{jb} )   (\bm{h}_{jb}|}}
  {Z_{ia,jb}}
\end{equation*}

previamente se ha utilizado ésta expresión para calcular el entrelazamiento entre excitaciones virtuales en un sistema
N-electrónico, además de varias medidas de la Teoría de la Información.
\footfullcite[]{Millan}
\footfullcite[]{JCP}.


\end{frame}

\section{Traza parcial de $\rho$}

\begin{frame}
  \frametitle{Traza parcial de $\rho$}
  \small
  Dado $\hat{O}$ un operador definido en el espacio de Hilbert $H$, en la mecánica cuántica de sistemas compuestos 
  con el espacio de Hilbert $H = H_a \otimes H_b$, se define la función \textit{función
  traza parcial}, tomado sobre el subsistema \textit{b}, como \footfullcite[]{RDM}
  
  \begin{equation*}
    Tr_b(\hat{O}) = \sum_{j=1}^{d_b} (\bm{\hat{I}_a} \otimes \langle b_j|) \hat{O} (\bm{\hat{I}_a} \otimes |b_j \rangle)
  \end{equation*}
  
  donde $|b_j\rangle$ cualquier base ortonormal para $H_b$ y $d_b= dim(H_b)$. The $\bm{\hat{I}_a}$ es el operador identidad en $H_a$.
\pause
\vfill
  Es importante notar que la anterior definición es equivalente a otra, que aparece muy frecuentemente en la literatura 
  \footfullcite[]{Nielsen_Chuang}
  
  \begin{equation*}
    Tr_b( |a \rangle \langle a' | \otimes | b\rangle \langle b' | ) = |a \rangle \langle a' | \otimes Tr (|b\rangle \langle b' |)
  \end{equation*}
  
  donde $|a \rangle$, $|a' \rangle$ $\epsilon$ $H_a$ y $|b\rangle$, $|b'\rangle$ $\epsilon$ $H_b$ son vectores genéricos en 
  los correspondientes espacios de Hilbert.

\end{frame}

\begin{frame}

\small
Se puede utilizar ambas definiciones dentro del formalismo de funciones de onda para definir una función de traza parcial 
para la matriz densidad dentro del formalismo de Propagadores de Polarización.
\vfill 
\pause
Si se considera a los elementos de $\rho$ correspondentes a dichas excitaciones ($i\rightarrow a, j \rightarrow b)$ como un subsistema 
(subsistema $A$), 
y a los elementos de $\rho$ correspondientes al resto de las excitaciones como otro subsistema (subsistema $B$), se puede
hallar la matriz densidad reducida $\rho_{ia,jb}$ haciendo la traza de $\rho$ con respecto al sub-sistema $B$. %\cite{Nielsen_Chuang}
\pause
\vfill
Para eso, se define al operador identidad del espacio de 
Fock que corresponde a las excitaciones $i\rightarrow a$ como un operador identidad en la posición $\{ i,a \}$  y 
cero en el resto de posiciones,

\begin{equation*}
  \bm{\hat{I}}_{ia,jb} = \left\{ 0, ..., \hat{I}_{ia}, 0,...,\hat{I}_{jb}, 0, ... \right\}
\end{equation*}

y el operador identidad que corresponde a las excitaciones $j\rightarrow b$ como un operador identidad 
en la posición $\{j,b\}$ y cero en las demás posiciones.
Lo cual significa que en las posiciones de las excitaciones $ia,jb$ los elementos matriciales de la matriz densidad
quedan invariantes.

\end{frame}


%\begin{frame}
%  Multiplicando $\rho$ a izquierda y derecha por $ \bm{\hat{I}}_{ia}$ y $ \bm{\hat{I}}_{jb}$

%  \begin{equation*}
%    \frac{e^{(\bm{\hat{I}}_{ia}| \bm{\widetilde{h}} ) (\bm{h}| E \hat{I} - \hat{H}_0 | \bm{\widetilde{h}} )   (\bm{h}| \bm{\hat{I}}_{jb})}}%
%	{Z}
%  \end{equation*}
%\end{frame}

\begin{frame}
  \small
A su vez, la traza sobre el subsistema $B$  se obtiene multiplicando a izquierda por 
los operadores $\bm{\hslash}_{p,q}$ formado por un operador \textit{hueco-partícula} y 
\textit{partícula-hueco} en el lugar de excitación $p\rightarrow q$ y cero en las restantes, y 
se suma sobre todos los $p,q$

%\begin{equation*}
%	Tr(\rho) =  \sum_{p,q} ( \bm{\hslash}_{p,q} | \rho  \bm{\widetilde{\hslash}_{p,q}} )   
%\end{equation*}

\begin{equation*}
	( \bm{\hslash}_{p,q} | = \left(\begin{matrix}
		0 \\
		\cdots \\
		(h_{pq}| \\
		\cdots \\
		0
	\end{matrix} \right)
\end{equation*}

Y a derecha por operadores filas del tipo

\begin{equation*}
	| \widetilde{\bm{\hslash}}_{p,q} ) = \begin{matrix}
		( 0 &\cdots& |h_{p,q})& \cdots& 0)
	\end{matrix}
\end{equation*}

donde se define a cada operador $h_{p,q}$ como un conjunto de operadores 
\textit{partícula-hueco} y \textit{hueco-partícula} entre los orbitales ocupados y desocupados $p,q$.

\begin{equation*}
	h_{p,q} = \{a^\dagger_p a_q, a^\dagger_p a_q \}
\end{equation*}

\end{frame}


\begin{frame}
  Por lo que, habiendo definido al sistema $B$ como aquel cuyas excitaciones son distintas de 
$i\rightarrow a, j \rightarrow b$  

\begin{equation*}
	Tr_B(\rho) = \sum_{p,q}  \left( (\bm{\hat{I}}_{ia}| \otimes| ( \bm{\hslash}_{p,q} | \right) \rho  \left( |\bm{\hat{I}}_{jb}) \otimes |\bm{\hslash}_{p,q}) \right)
\end{equation*}

lo que es equivalente a 
\begin{equation*}
  \rho_{ia,jb} = e^{ { (\bm{h}_{ia}| E \hat{I} - \hat{H}_0 | 
	\bm{\widetilde{h}}_{ia} ) }} \otimes
	Tr(e^{| {\bm{\widetilde{h}}_B ) (\bm{h}_B| E \hat{I} - \hat{H}_0 | 
	\bm{\widetilde{h}}_B ) )   (\bm{h}_B|}} )\times \frac{1}{Z}
\end{equation*}

\end{frame}

\section{Denominador de $\rho$}
\small
\begin{frame}

Ya que el exponente de la función partición utiliza la inversa del propagador, 
que contiene el \textit{manifold} de operadores $\bm{h}$, y que se pueden considerar
excitaciones de cada subsistema de manera separada, se propone  expresar la 
función partición del sistema como un producto de las funciones partición del sistema

\begin{equation*}
  Z = Z_{ia,jb}  Z_B
\end{equation*}

%\begin{equation*}
%  | \bm{\widetilde{h}}_{ia} ) (\bm{h_{ia}}| E \hat{I} - \hat{H}_0 | \bm{\widetilde{h}}_{jb} )   (\bm{h_{jb}}| \otimes
%  | \bm{\widetilde{h}}_B ) (\bm{h_B}| E \hat{I} - \hat{H}_0 | \bm{\widetilde{h}}_B )   (\bm{h_B}|
%\end{equation*} 

%la exponencial de la última ecuación puede ser expresada como 

%\begin{eqnarray*}
%   & & e^{| \bm{\widetilde{h}}_{ia} ) (\bm{h_{ia}}| E \hat{I} - \hat{H}_0 | \bm{\widetilde{h}}_{jb} )   (\bm{h_{jb}}| \otimes
%  | \bm{\widetilde{h}}_B ) (\bm{h_B}| E \hat{I} - \hat{H}_0 | \bm{\widetilde{h}}_B )   (\bm{h_B}| } \\
%  &=& e^{| \bm{\widetilde{h}}_{ia} ) (\bm{h_{ia}}| E \hat{I} - \hat{H}_0 | \bm{\widetilde{h}}_{jb} )   (\bm{h_{jb}}|} \times
%  e^{| \bm{\widetilde{h}}_B ) (\bm{h_B}| E \hat{I} - \hat{H}_0 | \bm{\widetilde{h}}_B )   (\bm{h_B}|} \\ 
%\end{eqnarray*}
\pause

Si se tiene en cuenta que la función de partición $Z$ representa la traza de 
la exponencial,  
como se puede observar en el artículo de Millán et al. \footfullcite[]{Millan}. 
Analizando la condición de la traza de una matriz densidad $\rho$ debe ser 1 

\pause
\begin{equation*}
	Tr \rho =  \sum_k \frac{(\bm{h}_k|e^{| \bm{\widetilde{h}} ) (\bm{h}| E \hat{I} - \hat{H}_0 | \bm{\widetilde{h}} )   
  (\bm{h}|} |\bm{\widetilde{h}}_k) } 
	{Z} = 1
\end{equation*}

\pause
la traza parcial de $\rho$ sobre el subsistema $B$ queda 

\begin{eqnarray*}
	Tr_B \rho &=& \frac{e^{| {\bm{\widetilde{h}}_{ia,jb} ) (\bm{h}_{ia,jb}| E \hat{I} - \hat{H}_0 | 
	\bm{\widetilde{h}}_{ia,jb} )   (\bm{h}_{ia,jb}|}}}{Z_{ia,jb}} \otimes \frac{Tr(\rho_B)}{Z_{B}} \\
	&=& \rho_{ia,jb} 
\end{eqnarray*}






%Cuando definimos a los subsistemas $A$ y $B$ como subsistemas de $\rho$, 
%debemos definir a la función partición para cada sistema, que va a contener 
%los caminos de excitación correspondientes a cada sistema.
\vfill
 

%\begin{equation*}
%  Tr (e^{  |\bm{\widetilde{h}}_{ia}) (\bm{h}_{ia}| E \hat{I} - \hat{H}_0 | %
%	\bm{\widetilde{h}}_{jb} ) (\bm{h}_{ia}| })
%\end{equation*}



\end{frame}

\section{Numerador}
\begin{frame}{Exponencial expresada como serie}
Sabiendo que se puede escribir la exponencial de un operador como una serie de potencias del operador


\begin{equation*}\label{exp_rho}
	e^{\hat{x}} = \hat{I} + \hat{x} + \frac{\hat{x}^2}{2} + \cdots
\end{equation*}

\pause

\begin{eqnarray*}\label{serie_rho}
	Tr_B (\rho) &=&  \rho_{ia,jb}  \sum_{p,q} ( \bm{\hslash}_{p,q} | \hat{I} +  | \bm{\widetilde{h}}_2 ) (\bm{h}_2| E \hat{I} - \hat{H}_0 |
	 \bm{\widetilde{h}}_2 )   (\bm{h}_2|  + \cdots  \widetilde{\bm{\hslash}}_{p,q} ) \times \frac{1}{Z_B} \\
	 &=& \rho_{ia,jb} \sum_{p,q} \left[ ( \bm{\hslash}_{p,q} |  \hat{I} \widetilde{\bm{\hslash}}_{p,q} ) 
	  + ( \bm{\hslash}_{p,q} | (\bm{h}_2| E \hat{I} - \hat{H}_0 |
	 \bm{\widetilde{h}}_2 )   (\bm{h}_2| \widetilde{\bm{\hslash}}_{p,q} ) + \cdots \right]  \times \frac{1}{Z_B}
\end{eqnarray*}

\end{frame}
\small
\begin{frame}{Término de órden cero}
  
  se evalúa el producto $( \bm{\hslash}_{p,q} |\hat{I} \bm{\widetilde{\hslash}}_{p,q} )$


\begin{eqnarray*}
	( \bm{\hslash}_{p,q} |\hat{I} \bm{\widetilde{\hslash}}_{p,q} ) &=& 
	\begin{pmatrix}
		0 \\\cdots\\
		(h_{pq}| \\
		\cdots \\
		0 \end{pmatrix}
 \times \begin{pmatrix}
	0 & \cdots & |h_{p,q})& \cdots& 0
\end{pmatrix} \\
	&=& \begin{pmatrix}
		0 & \cdots &  \cdots  \\
		\vdots & (h_{pq} | h_{pq}) & \vdots \\
		\vdots & 0 & \cdots
		\end{pmatrix}
\end{eqnarray*}

por lo que se obtiene una matriz con todos sus elementos cero, menos en el elemento $p,q$, 
con un valor $\langle0| [h^\dagger_{p,q}, h_{p,q}] |0\rangle$. 
Como está definido, los elementos $p,q$ de $h_{p,q}$ son los orbitales ocupados y desocupados, 
diferentes de $i,j$ y $a,b$.
\end{frame}

\begin{frame}
  El producto interno $(h_{pq} | h_{pq})$ está definido como 

  \begin{eqnarray*}
    (h_{pq} | h_{pq}) &=& \begin{pmatrix}
      (a^\dagger_p a_q| a^\dagger_p a_q) &  (a^\dagger_p a_q| a^\dagger_q a_p) \\
      (a^\dagger_q a_p| a^\dagger_p a_q)  & (a^\dagger_q a_p| a^\dagger_q a_p)
    \end{pmatrix} 
    = \begin{pmatrix}
      1 &  0 \\
      0 &  -1
      \end{pmatrix}
  \end{eqnarray*}

  Por lo que hacer la sumatoria sobre todos los orbitales $p,q$ perteneciantes al subsistema  
  $\sum_{p,q} (\bm{\hslash}_{p,q} |\hat{I} \bm{\widetilde{\hslash}}_{p,q})$ es igual a 0.
  
\end{frame}

\small

\small
\begin{frame}{Término de Primer Orden}
El término a analizar es 

\begin{equation*}
  ( \bm{\hslash}_{p,q} | (\bm{h}_2| E \hat{I} - \hat{H}_0 |
	 \bm{\widetilde{h}}_2 )   (\bm{h}_2| \widetilde{\bm{\hslash}}_{p,q} )
\end{equation*}
\pause
El término de la izquierda $( \bm{\hslash}_{p,q} | \bm{\widetilde{h}}_2 )$ está compuesto por

\begin{eqnarray*}
	( \bm{\hslash}_{p,q} | \bm{\widetilde{h}}_2 ) &=& \begin{pmatrix}
		
		0 \\
		\vdots \\
		(h_{pq}| \\
		\vdots \\
		0
	\end{pmatrix}
	\times \begin{pmatrix}
		
		 |h_{i'a'}) & |h_{j',b'}) & \cdots  & \cdots
	\end{pmatrix} \end{eqnarray*}

  
\end{frame}

\small
\begin{frame}
  por ejemplo, para $( \bm{\hslash}_{i',j'} | $, se tiene

\begin{eqnarray*} \label{hslash_qp}
	( \bm{\hslash}_{p,q} | \bm{\widetilde{h}}_2 )
		 &=& \begin{pmatrix}
		(h_{i',a'}) \\
		\vdots \\
		0 \\
		\vdots \\
		0
	\end{pmatrix}  \times  \begin{pmatrix}
		|h_{i'a'}) & \cdots & |h_{j',b'}) &  \cdots
	\end{pmatrix}            \\
     & = & \begin{pmatrix}
		 (h_{i',a'} | h_{i',a'}) & 0 & \cdots \\
		 0 & \vdots & \vdots \\
		 \vdots & \vdots & \vdots 
	 \end{pmatrix} 
\end{eqnarray*}

\end{frame}

\begin{frame}
  mientras que a derechas, se debe resolver el producto de matrices 
$( \bm{h}_2 | \widetilde{\bm{\hslash}}_{p,q})$

\begin{eqnarray*}\label{hslash_pq}
	( \bm{\hslash}_{p,q} | \bm{\widetilde{h}}_2 )
		 &=& \begin{pmatrix}
		(h_{i',a'}| \\
		 \vdots \\
		(h_{j',b'}| \\
		
		\vdots 
	\end{pmatrix}  \times  \begin{pmatrix}
		|h_{i'a'}) & 0 &  \cdots
	\end{pmatrix}            \\
     & = & \begin{pmatrix}
		 (h_{i',a'} | h_{i',a'}) & 0 & \cdots \\
		 0 & \vdots & \vdots \\
		 \vdots & \vdots & \vdots 
	 \end{pmatrix} 
\end{eqnarray*}

\end{frame}
\begin{frame}
  y a la matriz central,$(\bm{h}_B| E \hat{I} - \hat{H}_0 | \bm{\widetilde{h}}_B )$ 
que es el la inversa del Propagador Principal, al multiplicar a izquiera y a derecha por las matrices \ref{hslash_pq} y \ref{hslash_qp},
solo será distinto de cero el elemento $i'\rightarrow a', j \rightarrow b'$ del Propagador Principal. 

\begin{eqnarray*}
	\begin{pmatrix}
		(h_{i',a'} | h_{i',a'}) & 0 & \cdots \\
		0 & \vdots & \vdots \\
		\vdots & \vdots & \vdots 
	\end{pmatrix} \otimes \begin{pmatrix}
		M_{i'a',i'a'} & M_{i'a',j' b'} &  \cdots \\
		M_{j' b',i'a'} & M_{j' b',j' b'} & \vdots \\
		\vdots & \vdots & \vdots
	\end{pmatrix} \otimes
	\begin{pmatrix}
		(h_{i',a'} | h_{i',a'}) & 0 & \cdots \\
		0 & \vdots & \vdots \\
		\vdots & \vdots & \vdots 
	\end{pmatrix} = M_{i'a',i'a'}
	\end{eqnarray*}

Por lo que, al hacer una sumatoria sobre todos los orbitales $p,q$ diferentes de $i,j$ y $a,b$, se obtiene

\begin{equation*}
	Tr (\bm{h}_B| E \hat{I} - \hat{H}_0 | \bm{\widetilde{h}}_B)  =  \sum_{p,q  \epsilon B} M_{pq,pq}
\end{equation*}
\end{frame}


%slide 5, SCF no es HF. A lo sumo hablar de un estado de Referencia, autoconsistente.
% slide 7, la expresión de la operador creación en el orbital ocupado i, no está bien. 
%no se crea un orbital sobre un ocupado, nunca... sinó que, tenemos ese operador, que luego 
%se "relaciona" con otros, y luego se aplica al estado de referencia
% No está muy claro el uso de las excitaciones, y de cuál es la diferencia entre
% la rho de propagadores y de estados, que es que la de propagadores contiene excitaciones
% entre estados, y permite evaluar las excitaciones y sus efectos sobre las propiedades
%mientras que en el método de funciones de onda, no se puede....
% finalmente, a la ecuación que quería llegar, no se llegó, sin embargo, 
% podría quedar expresada

% rho = e^M_{ia,jb} \times 1/Z_{[0]}

%de hecho que así está en el paper del 2018. Tuve que haber sido más despierto
%y al no resolver lo que me proponía, tuve que haberme preguntado si estaba bien
%el objetivo del trabajo

\end{document}