%%%%%%%%%%%%%%%%%%%%%%%%%%%%%%%%%%%%%%%%%
% fphw Assignment
% LaTeX Template
% Version 1.0 (27/04/2019)
%
% This template originates from:
% https://www.LaTeXTemplates.com
%
% Authors:
% Class by Felipe Portales-Oliva (f.portales.oliva@gmail.com) with template 
% content and modifications by Vel (vel@LaTeXTemplates.com)
%
% Template (this file) License:
% CC BY-NC-SA 3.0 (http://creativecommons.org/licenses/by-nc-sa/3.0/)
%
%%%%%%%%%%%%%%%%%%%%%%%%%%%%%%%%%%%%%%%%%

%----------------------------------------------------------------------------------------
%	PACKAGES AND OTHER DOCUMENT CONFIGURATIONS
%----------------------------------------------------------------------------------------

\documentclass[
	12pt, % Default font size, values between 10pt-12pt are allowed
	%letterpaper, % Uncomment for US letter paper size
	%spanish, % Uncomment for Spanish
]{fphw}

% Template-specific packages
\usepackage[utf8]{inputenc} % Required for inputting international characters
\usepackage[T1]{fontenc} % Output font encoding for international characters
\usepackage{mathpazo} % Use the Palatino font

\usepackage{graphicx} % Required for including images

\usepackage{booktabs} % Required for better horizontal rules in tables

\usepackage{listings} % Required for insertion of code
\usepackage[utf8]{inputenc}
\usepackage{amsmath}
\usepackage{amsfonts}
\usepackage{amsthm}
\usepackage{amssymb}
\usepackage{mathrsfs}
\usepackage{enumitem}
\usepackage{physics}
\usepackage{bm}
\usepackage{enumerate} % To modify the enumerate environment
\usepackage{hyperref}
\usepackage[backend=biber,style=numeric,sorting=none]{biblatex}
\usepackage{caption}
\addbibresource{refs.bib}
\captionsetup[figure]{font=footnotesize,labelfont=footnotesize}

%----------------------------------------------------------------------------------------
%	ASSIGNMENT INFORMATION
%----------------------------------------------------------------------------------------

\title{Notas acerca del CLOPPAenta nglement} % Assignment title

\author{Daniel F. E. Bajac} % Student name

\date{6 de julio del 2022} % Due date

\institute{Universidad Nacional del Nordeste \\ Departamento de Física} % Institute or school name

%\class{Propagadores de Polarización} % Course or class name

\professor{Dr. Gustavo A. Aucar } % Professor or teacher in charge of the assignment

%----------------------------------------------------------------------------------------

\begin{document}

\maketitle % Output the assignment title, created automatically using the information in the custom commands above

%----------------------------------------------------------------------------------------
%	ASSIGNMENT CONTENT
%----------------------------------------------------------------------------------------

\section*{Caminos de acoplamiento en C$_2$H$_6$}

\subsection*{Acoplamiento utilizando todos los orbitales moleculares ocupados y virtuales}
Se analiza utilizando el programa \textit{pyPPE} los diferentes mecanismos que contribuyen al acoplamiento entre espines
nucleares J$_{ia,jb}$  a saber, los dependientes del espín nuclear Fermi-Contact (FC) y Spin Dependent (SD) y el 
independiente del espín electrónico Paramagnetic Spin Orbital (PSO), variando el ángulo diedro entre dos hidrógenos 
separados por tres enlaces. Para dicho cálculo se utilizó la aproximación CLOPPA  \cite{Cloppa} 
con orbitales moleculares previamente 
 localizados con las transformaciones de Pipek-Mezey, implementados en el código \textit{pySCF} 

\begin{figure}[h]
	\centering
	\includegraphics*[scale=0.4]{ssc_mechanims_C2H6_ccpvdz}	
	\caption{Mecanismos que contribuyen al acoplamiento J(H-H) del C$_2$H$_6$, utilizando el método CLOPPA}
	
\end{figure}

donde se observa, en principio, que el acoplamiento entre los hidrógenos cumple con la regla de Karplus, 
la regla empírica que relaciona el acoplamiento entre 
que la contribución SD y PSO son pequeñas comparadas con el acoplamiento total, 
y que el mecanismo FC tiene la mayor contribución al mismo. Es importante mostrar que los resultados obtenidos son similares a los
encontrados con el programa Dalton, utilizando orbitales moleculares canónicos.

\begin{figure}[h]
	\centering
	\includegraphics*[scale=0.5]{mecanismos_etano_dalton}	
	\caption{Mecanismos que contribuyen al acoplamiento J(H-H) del C$_2$H$_6$, utilizando orbitales moleculares canónicos}
\end{figure}. 

\subsection*{Eligiendo caminos de acoplamiento con diferentes LMOs ocupados y todos los virtuales}

En ésta sección se muestran los valores del acoplamiento J utilizando diferentes LMOs ocupados y todos los LMOs virtuales.
Se busca cuáles son los orbitales moleculares $i,j$ que más contribuyen a la propiedad $J(H-H)_{ia,jb}$, siendo
$i,j (a,b)$ orbitales moleculares ocupados (desocupados). 

Para dicho fin, es menester detallar de qué manera cambian los valores de $J_{ia,jb} (H-H)$ si los LMOs se encuentran centrados
en la misma zona, y por ende se van a superponer, o si se encuentran separados espacialmente. Éstos son los 
mecanismos hiperconjuntivos posibles en el cálculo de $J_{ia,jb}$

\begin{figure}[h]
	\centering
	\includegraphics*[scale=0.3]{coupling_types}	
	\caption{Mecanismos hiperconjuntivos: (a)local-local hyperconjugative (LLH),(b) double-vicinal hyper-
	conjugative (DVH); (c) double-local hyperconjugative (DLH); (d)
	local-vicinal hyperconjugative (LVH)}
\end{figure}. 



\printbibliography

\end{document}
